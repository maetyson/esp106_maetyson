% Options for packages loaded elsewhere
\PassOptionsToPackage{unicode}{hyperref}
\PassOptionsToPackage{hyphens}{url}
\documentclass[
]{article}
\usepackage{xcolor}
\usepackage[margin=1in]{geometry}
\usepackage{amsmath,amssymb}
\setcounter{secnumdepth}{-\maxdimen} % remove section numbering
\usepackage{iftex}
\ifPDFTeX
  \usepackage[T1]{fontenc}
  \usepackage[utf8]{inputenc}
  \usepackage{textcomp} % provide euro and other symbols
\else % if luatex or xetex
  \usepackage{unicode-math} % this also loads fontspec
  \defaultfontfeatures{Scale=MatchLowercase}
  \defaultfontfeatures[\rmfamily]{Ligatures=TeX,Scale=1}
\fi
\usepackage{lmodern}
\ifPDFTeX\else
  % xetex/luatex font selection
\fi
% Use upquote if available, for straight quotes in verbatim environments
\IfFileExists{upquote.sty}{\usepackage{upquote}}{}
\IfFileExists{microtype.sty}{% use microtype if available
  \usepackage[]{microtype}
  \UseMicrotypeSet[protrusion]{basicmath} % disable protrusion for tt fonts
}{}
\makeatletter
\@ifundefined{KOMAClassName}{% if non-KOMA class
  \IfFileExists{parskip.sty}{%
    \usepackage{parskip}
  }{% else
    \setlength{\parindent}{0pt}
    \setlength{\parskip}{6pt plus 2pt minus 1pt}}
}{% if KOMA class
  \KOMAoptions{parskip=half}}
\makeatother
\usepackage{graphicx}
\makeatletter
\newsavebox\pandoc@box
\newcommand*\pandocbounded[1]{% scales image to fit in text height/width
  \sbox\pandoc@box{#1}%
  \Gscale@div\@tempa{\textheight}{\dimexpr\ht\pandoc@box+\dp\pandoc@box\relax}%
  \Gscale@div\@tempb{\linewidth}{\wd\pandoc@box}%
  \ifdim\@tempb\p@<\@tempa\p@\let\@tempa\@tempb\fi% select the smaller of both
  \ifdim\@tempa\p@<\p@\scalebox{\@tempa}{\usebox\pandoc@box}%
  \else\usebox{\pandoc@box}%
  \fi%
}
% Set default figure placement to htbp
\def\fps@figure{htbp}
\makeatother
\setlength{\emergencystretch}{3em} % prevent overfull lines
\providecommand{\tightlist}{%
  \setlength{\itemsep}{0pt}\setlength{\parskip}{0pt}}
\usepackage{bookmark}
\IfFileExists{xurl.sty}{\usepackage{xurl}}{} % add URL line breaks if available
\urlstyle{same}
\hypersetup{
  pdftitle={Lab 5},
  hidelinks,
  pdfcreator={LaTeX via pandoc}}

\title{Lab 5}
\author{}
\date{\vspace{-2.5em}}

\begin{document}
\maketitle

\begin{verbatim}
## Loading required package: knitr
\end{verbatim}

\begin{verbatim}
## Warning: package 'knitr' was built under R version 4.5.2
\end{verbatim}

\subsection{Lab 5}\label{lab-5}

\subsubsection{Due Tuesday Feb4th - Recommended to complete this before
starting the
midterm}\label{due-tuesday-feb4th---recommended-to-complete-this-before-starting-the-midterm}

This lab we will look at some data from the plastic trash piced up
during clean-up events around the world. I took this dataset from the
Tidy Tuesday website. You can read the documentation
\href{https://github.com/rfordatascience/tidytuesday/blob/master/data/2021/2021-01-26/readme.md}{here},
including the references and description of the different column names.

I have done some pre-processing of the data for you for this lab, to
create two more easy-to-use dataframes.

First read in the countrytotals.csv data frame

\begin{verbatim}
## Warning: package 'ggplot2' was built under R version 4.5.2
\end{verbatim}

Have a look at the data frame. Then column ``total'' gives the total
number of pieces of plastic picked up in that country in 2020. The
columns ``num\_events'' and ``volunteers'' give the number of trash
pick-up events and the number of volunteers in that country. We are
going to use this to investigate where the plastic trash problem is
worst.

\begin{enumerate}
\def\labelenumi{\arabic{enumi}.}
\tightlist
\item
  What 5 countries had the worst plastic problem as measured by the
  number of pieces of trash picked up?
\end{enumerate}

\textbf{Answer:}

\begin{enumerate}
\def\labelenumi{\arabic{enumi}.}
\setcounter{enumi}{1}
\tightlist
\item
  Make a plot showing the distribution of volunteers across countries
\end{enumerate}

\pandocbounded{\includegraphics[keepaspectratio]{lab5_files/figure-latex/question2-1.pdf}}

\begin{enumerate}
\def\labelenumi{\arabic{enumi}.}
\setcounter{enumi}{2}
\tightlist
\item
  Notice that there is a lot of variation across countries in the number
  of volunteers involved in trash pickup. What problem might that cause
  for the interpretation of your answer to question 1?
\end{enumerate}

\textbf{Answer:} \#\# if one countries has a large amount of volunteers
they may pick up more trash, they will collect a lot of trash. Another
country that has way less volunteers makes it seems like there is a lot
less trash to be collected 4. Add a column to the data frame creating a
variable that should be more closely related to the presence of plastic
pollution in the country

\begin{enumerate}
\def\labelenumi{\arabic{enumi}.}
\setcounter{enumi}{4}
\tightlist
\item
  What 5 countries have the worst plastic pollution, as measured by this
  new variable?
\end{enumerate}

\textbf{Answer:}

Now we will make a plot of the variation in the types of trash and how
it differs around the world. Read in the continenttypes.csv data frame.
This gives the breakdown of the different types of plastic collected on
each continet in 2020 and the total number of pick up events.

\begin{enumerate}
\def\labelenumi{\arabic{enumi}.}
\setcounter{enumi}{5}
\item
  Add a column to this data frame with a variable that captures the
  existence of different types of plastic trash, controlling for the
  intensity of the pick-up effort in different continent
\item
  Make a plot using ggplot showing both the total amount and
  distribution of types of plastic picked up in each continent in the
  average pick-up event.
\end{enumerate}

Hint: Check out options in the \href{https://www.r-graph-gallery.com}{R
graph gallery}

\pandocbounded{\includegraphics[keepaspectratio]{lab5_files/figure-latex/question7-1.pdf}}

\begin{enumerate}
\def\labelenumi{\arabic{enumi}.}
\setcounter{enumi}{7}
\tightlist
\item
  Try uploading your R markdown file and plots to your Git Hub
  repository. Upload your Rmd and knitted PDF to Canvas
\end{enumerate}

\end{document}
